%%%%%%%%%%%%%%%%%%%%%%%%%%%%%%%%%%%%%%%%%
% Beamer Presentation
% LaTeX Template
% Version 2.0 (March 8, 2022)
%
% This template originates from:
% https://www.LaTeXTemplates.com
%
% Author:
% Vel (vel@latextemplates.com)
%
% License:
% CC BY-NC-SA 4.0 (https://creativecommons.org/licenses/by-nc-sa/4.0/)
%
%%%%%%%%%%%%%%%%%%%%%%%%%%%%%%%%%%%%%%%%%

%----------------------------------------------------------------------------------------
%	PACKAGES AND OTHER DOCUMENT CONFIGURATIONS
%----------------------------------------------------------------------------------------

\documentclass[
	11pt, % Set the default font size, options include: 8pt, 9pt, 10pt, 11pt, 12pt, 14pt, 17pt, 20pt
	%t, % Uncomment to vertically align all slide content to the top of the slide, rather than the default centered
	%aspectratio=169, % Uncomment to set the aspect ratio to a 16:9 ratio which matches the aspect ratio of 1080p and 4K screens and projectors
]{beamer}

\graphicspath{{Images/}{./}} % Specifies where to look for included images (trailing slash required)

\usepackage{booktabs} % Allows the use of \toprule, \midrule and \bottomrule for better rules in tables

%----------------------------------------------------------------------------------------
%	SELECT LAYOUT THEME
%----------------------------------------------------------------------------------------

% Beamer comes with a number of default layout themes which change the colors and layouts of slides. Below is a list of all themes available, uncomment each in turn to see what they look like.

%\usetheme{default}
%\usetheme{AnnArbor}
%\usetheme{Antibes}
%\usetheme{Bergen}
%\usetheme{Berkeley}
%\usetheme{Berlin}
%\usetheme{Boadilla}
%\usetheme{CambridgeUS}
%\usetheme{Copenhagen}
%\usetheme{Darmstadt}
%\usetheme{Dresden}
%\usetheme{Frankfurt}
%\usetheme{Goettingen}
%\usetheme{Hannover}
%\usetheme{Ilmenau}
%\usetheme{JuanLesPins}
%\usetheme{Luebeck}
\usetheme{Madrid}
%\usetheme{Malmoe}
%\usetheme{Marburg}
%\usetheme{Montpellier}
%\usetheme{PaloAlto}
%\usetheme{Pittsburgh}
%\usetheme{Rochester}
%\usetheme{Singapore}
%\usetheme{Szeged}
%\usetheme{Warsaw}

%----------------------------------------------------------------------------------------
%	SELECT COLOR THEME
%----------------------------------------------------------------------------------------

% Beamer comes with a number of color themes that can be applied to any layout theme to change its colors. Uncomment each of these in turn to see how they change the colors of your selected layout theme.

%\usecolortheme{albatross}
%\usecolortheme{beaver}
%\usecolortheme{beetle}
%\usecolortheme{crane}
%\usecolortheme{dolphin}
%\usecolortheme{dove} *
%\usecolortheme{fly}
%\usecolortheme{lily}
%\usecolortheme{monarca}
%\usecolortheme{seagull}
%\usecolortheme{seahorse} **
\usecolortheme{spruce}
%\usecolortheme{whale}
%\usecolortheme{wolverine}

%----------------------------------------------------------------------------------------
%	SELECT FONT THEME & FONTS
%----------------------------------------------------------------------------------------

% Beamer comes with several font themes to easily change the fonts used in various parts of the presentation. Review the comments beside each one to decide if you would like to use it. Note that additional options can be specified for several of these font themes, consult the beamer documentation for more information.

\usefonttheme{default} % Typeset using the default sans serif font
%\usefonttheme{serif} % Typeset using the default serif font (make sure a sans font isn't being set as the default font if you use this option!)
%\usefonttheme{structurebold} % Typeset important structure text (titles, headlines, footlines, sidebar, etc) in bold
%\usefonttheme{structureitalicserif} % Typeset important structure text (titles, headlines, footlines, sidebar, etc) in italic serif
%\usefonttheme{structuresmallcapsserif} % Typeset important structure text (titles, headlines, footlines, sidebar, etc) in small caps serif

%------------------------------------------------

%\usepackage{mathptmx} % Use the Times font for serif text
\usepackage{palatino} % Use the Palatino font for serif text

%\usepackage{helvet} % Use the Helvetica font for sans serif text
\usepackage[default]{opensans} % Use the Open Sans font for sans serif text
%\usepackage[default]{FiraSans} % Use the Fira Sans font for sans serif text
%\usepackage[default]{lato} % Use the Lato font for sans serif text

%----------------------------------------------------------------------------------------
%	SELECT INNER THEME
%----------------------------------------------------------------------------------------

% Inner themes change the styling of internal slide elements, for example: bullet points, blocks, bibliography entries, title pages, theorems, etc. Uncomment each theme in turn to see what changes it makes to your presentation.

%\useinnertheme{default}
\useinnertheme{circles}
%\useinnertheme{rectangles}
%\useinnertheme{rounded}
%\useinnertheme{inmargin}

%----------------------------------------------------------------------------------------
%	SELECT OUTER THEME
%----------------------------------------------------------------------------------------

% Outer themes change the overall layout of slides, such as: header and footer lines, sidebars and slide titles. Uncomment each theme in turn to see what changes it makes to your presentation.

%\useoutertheme{default}
%\useoutertheme{infolines}
%\useoutertheme{miniframes}
%\useoutertheme{smoothbars}
%\useoutertheme{sidebar}
%\useoutertheme{split}
%\useoutertheme{shadow}
%\useoutertheme{tree}
%\useoutertheme{smoothtree}

%\setbeamertemplate{footline} % Uncomment this line to remove the footer line in all slides
%\setbeamertemplate{footline}[page number] % Uncomment this line to replace the footer line in all slides with a simple slide count

%\setbeamertemplate{navigation symbols}{} % Uncomment this line to remove the navigation symbols from the bottom of all slides

%----------------------------------------------------------------------------------------
%	PRESENTATION INFORMATION
%----------------------------------------------------------------------------------------

\title[AI Agent]{In The Name of Allah} % The short title in the optional parameter appears at the bottom of every slide, the full title in the main parameter is only on the title page

\subtitle{AI Agent: Prof Recommender} % Presentation subtitle, remove this command if a subtitle isn't required

\author[Amirhossein Alian]{Prof: Dr Hamidreza Hosseinkhani} % Presenter name(s), the optional parameter can contain a shortened version to appear on the bottom of every slide, while the main parameter will appear on the title slide

\institute[]{Sharif University of Technology \\ \smallskip \textit{}} % Your institution, the optional parameter can be used for the institution shorthand and will appear on the bottom of every slide after author names, while the required parameter is used on the title slide and can include your email address or additional information on separate lines

\date[\today]{Amirhossein Alian \\ \today} % Presentation date or conference/meeting name, the optional parameter can contain a shortened version to appear on the bottom of every slide, while the required parameter value is output to the title slide

%----------------------------------------------------------------------------------------

\begin{document}

%----------------------------------------------------------------------------------------
%	TITLE SLIDE
%----------------------------------------------------------------------------------------

\begin{frame}
	\titlepage % Output the title slide, automatically created using the text entered in the PRESENTATION INFORMATION block above
\end{frame}

%----------------------------------------------------------------------------------------
%	TABLE OF CONTENTS SLIDE
%----------------------------------------------------------------------------------------

% The table of contents outputs the sections and subsections that appear in your presentation, specified with the standard \section and \subsection commands. You may either display all sections and subsections on one slide with \tableofcontents, or display each section at a time on subsequent slides with \tableofcontents[pausesections]. The latter is useful if you want to step through each section and mention what you will discuss.

\begin{frame}
	\frametitle{Presentation Overview} % Slide title, remove this command for no title
	
	\tableofcontents % Output the table of contents (all sections on one slide)
	%\tableofcontents[pausesections] % Output the table of contents (break sections up across separate slides)
\end{frame}

%----------------------------------------------------------------------------------------
%	PRESENTATION BODY SLIDES
%----------------------------------------------------------------------------------------

\section{Class Diagram} % Sections are added in order to organize your presentation into discrete blocks, all sections and subsections are automatically output to the table of contents as an overview of the talk but NOT output in the presentation as separate slides

%------------------------------------------------

\begin{frame}
	\frametitle{UML (Structured) Class Diagram}
	
	\begin{figure}
		\includegraphics[width=0.9\linewidth]{arch2.pdf}
%				\caption{Creodocs logo.}
	\end{figure}
\end{frame}

%------------------------------------------------

\section{Neo4j}

\begin{frame}
	\frametitle{Neo4j as Knowledge Base}
\end{frame}

%------------------------------------------------

\section{OpenAI and Ollama as LLMs}

\subsection{OpenAI API}

\begin{frame}
	\frametitle{OpenAI API Query Format}
	\framesubtitle{gpt-4} % Optional subtitle
	
	\begin{itemize}
		\item \{
		\begin{itemize}
			\item "model": "gpt-4",
			\item "messages": [
				\begin{itemize}
					\item 	    \{"role": "system", "content": " PROMPT\_1 "\},
										\item 	    \{"role": "user", "content": " PROMPT\_2 "\}
				\end{itemize}
			\item ],
			\item   "max\_tokens": 200
		\end{itemize}
		\item \}
	\end{itemize}
	
	\bigskip % Vertical whitespace
	
	\begin{enumerate}
		\item PROMPT\_1 = "You are a Cypher query generator. Only provide Cypher queries in a single line, without any line breaks or extra formatting."
		\item PROMPT\_2 = "i want to take Computer Vision course this term, Which professor you suggest?"
	\end{enumerate}
\end{frame}

%------------------------------------------------

\subsection{Ollama API}

\begin{frame}
	\frametitle{Ollama API Query Format}
	\framesubtitle{codellama:7b} % Optional subtitle
	
	\begin{itemize}
		\item \{
		\begin{itemize}
			\item "model": "codellama",
			\item "prompt": " PROMPT\_1 ",
			\item  "stream": "false"
		\end{itemize}
		\item \}
	\end{itemize}
	
	\bigskip % Vertical whitespace
	
	\begin{enumerate}
		\item PROMPT\_1 = "USER\_REQUEST='\%s'   Please create a Cypher query based on USER\_REQUEST for a graph in Neo4j where nodes represent Professors."
	\end{enumerate}
\end{frame}

%------------------------------------------------

\subsection{Extra Query}

\begin{frame}
	\frametitle{Explain Graph nodes and their properties for LLMs}
		
	\begin{enumerate}
		\item Graph nodes represent Professors. Each Professor node has the following properties: professor\_id (which is unique and should be used to identify the professor), first\_name, last\_name, email, department, teaching, collaboration\_start\_year, research\_interests, and publications\_file. The professor\_id should be used as the unique identifier for each node, and the query should create nodes for Professors with the given properties.
	IMPORTANT: When filtering based on the request, the query should not require an exact match for fields like 'teaching' or 'research\_interests', but should instead search for partial matches (e.g., using the CONTAINS operator) for the keyword. Additionally, the results should be sorted by collaboration\_start\_year in ascending order (i.e., from the smallest number upward)
	\end{enumerate}
\end{frame}

%------------------------------------------------

\section{Input/Output}

\begin{frame}
	\frametitle{Definitions \& Examples}
	
	\begin{definition}
		A \alert{prime number} is a number that has exactly two divisors.
	\end{definition}
	
	\smallskip % Vertical whitespace
	
	\begin{example}
		\begin{itemize}
			\item 2 is prime (two divisors: 1 and 2).
			\item 3 is prime (two divisors: 1 and 3).
			\item 4 is not prime (\alert{three} divisors: 1, 2, and 4).
		\end{itemize}
	\end{example}
	
	\smallskip % Vertical whitespace
	
	You can also use the \texttt{theorem}, \texttt{lemma}, \texttt{proof} and \texttt{corollary} environments.
\end{frame}

%----------------------------------------------------------------------------------------
%	ACKNOWLEDGMENTS SLIDE
%----------------------------------------------------------------------------------------

\begin{frame}
	\frametitle{Acknowledgements}
	
	\begin{columns}[t] % The "c" option specifies centered vertical alignment while the "t" option is used for top vertical alignment
		\begin{column}{0.45\textwidth} % Left column width
			\textbf{Smith Lab}
			\begin{itemize}
				\item Alice Smith
				\item Devon Brown
			\end{itemize}
			\textbf{Cook Lab}
			\begin{itemize}
				\item Margaret
				\item Jennifer
				\item Yuan
			\end{itemize}
		\end{column}		
		\begin{column}{0.5\textwidth} % Right column width
			\textbf{Funding}
			\begin{itemize}
				\item British Royal Navy
				\item Norwegian Government
			\end{itemize}
		\end{column}
	\end{columns}
\end{frame}

%----------------------------------------------------------------------------------------
%	CLOSING SLIDE
%----------------------------------------------------------------------------------------

\begin{frame}[plain] % The optional argument 'plain' hides the headline and footline
	\begin{center}
		{\Huge The End}
		
		\bigskip\bigskip % Vertical whitespace
		
		{\LARGE Questions? Comments?}
	\end{center}
\end{frame}

%----------------------------------------------------------------------------------------

\end{document} 